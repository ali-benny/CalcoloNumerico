{\color{bblue}\subsection{Osservazioni}}
Osserviamo i risultati ottenuti su un'immagine scelta casualmente del set creato e sulle due immagini 
 fotografiche aggiunte.

Ricordiamo che più è alto il valore del PSNR maggiore sarà la vicinanza dell'immagine corrotta 
all'immagine originale. 

{\color{bblue}\subsubsection{Analisi immagine geometrica}}
Analizziamo l'immagine img8.png al variare del valore $\sigma$ con \verb|noise| fissato a 0.02:

\begin{figure}[H]
    \centering
    \begin{subfigure}{0.6\textwidth}
        \centering
        \includegraphics[width=\textwidth]{imgRel/img8corrotto/img8corrotta5x5.png}
        \caption{Img8 corrotta con $\sigma = 0.5$ dimensione $5 \times 5$}
        \label{fig:8corrotto5}
    \end{subfigure}
    \begin{subfigure}{0.6\textwidth}
        \centering
        \includegraphics[width=\textwidth]{imgRel/img8corrotto/img8corrotta7x7.png}
        \caption{Img8 corrotta con $\sigma = 1$ dimensione $7\times 7$}
        \label{fig:8corrotto7}
    \end{subfigure}
    \begin{subfigure}{0.6\textwidth}
        \centering
        \includegraphics[width=\textwidth]{imgRel/img8corrotto/img8corrotta9x9.png}
        \caption{Img8 corrotta con $\sigma = 1.3$ dimensione $9 \times 9$}
        \label{fig:8corrotto9}
    \end{subfigure}
    \caption{Immagine geometrica corrotta al variare di $\sigma$}
    \label{fig:8corrotto}
\end{figure}
Le figure di sinistra rappresentano l'immagine originale, invece a destra sono riportate le immagini corrotte 
con i rispettivi valori di PSNR e MSE. 
Notiamo che all'aumentare delle dimensioni di sigma il valore di PSNR diminuisce: ciò denota un peggioramento 
della qualità dell'immagine. Infatti le immagini subiscono un affievolimento dell'intensità della scala dei 
colori e i contorni delle varie figure geometriche perdono di fermezza. 

{\color{bblue}\subsubsection{Analisi immagini fotografiche}}
Analizziamo le immagini fotografiche al variare del valore $\sigma$ con \verb|noise| fissato a 0.02:
\begin{figure}[H]
    \centering
    \begin{subfigure}{0.67\textwidth}
        \centering
        \includegraphics[width=\textwidth]{confrontiPuntoUno/5-0.5-0.02.png}
        \caption{immagini corrotte con $\sigma = 0.5$ dimensione $5 \times 5$}
        \label{fig:fotogrCorrotte5x5}
    \end{subfigure}
    \begin{subfigure}{0.67\textwidth}
        \centering
        \includegraphics[width=\textwidth]{confrontiPuntoUno/7-1-0.02.png}
        \caption{immagini corrotte con $\sigma = 1$ dimensione $7 \times 7$}
        \label{fig:fotogrCorrotte7x7}
    \end{subfigure}
    \begin{subfigure}{0.67\textwidth}
        \centering
        \includegraphics[width=\textwidth]{confrontiPuntoUno/9-1.3-0.02.png}
        \caption{immagini corrotte con $\sigma = 1.3$ dimensione $9 \times 9$}
        \label{fig:fotogrCorrotte9x9}
    \end{subfigure}
    \caption{Immagini fotografiche corrotte al variare di $\sigma$}
    \label{fig:fotogrCorrotte}
\end{figure}

Si nota un'altra volta che all'aumentare delle dimensioni di $\sigma$ diminuisce il PSNR e l'immagine perde di 
incisività (nitidezza).
In ogni caso nelle versioni corrotte, benché risultino visivamente peggiori, si riesce ancora a ben distinguere il 
soggetto in primo piano, anche se sfocato, in tutte le immagini. 
