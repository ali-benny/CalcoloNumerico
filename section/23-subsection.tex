\subsection{Metodo del Gradiente e Metodo del Gradiente Coniugato a confronto}

Notiamo che tra i due metodi che il primo ci da come risultato delle immagini con un PSNR 
definitivamente più alto rispetto al secondo, le immagini sono qualitativamente più simili 
alle immagini originali. 

\begin{figure}[H]
    \centering
    \includegraphics[width=\textwidth]{output/MGCvsMG-enph.png}
    \label{fig:MGCvsMG}
    \caption{Gradiente immagine data.camera()}
\end{figure}

Per quanto riguarda l'immagine fotografica pugile.png, abbiamo riscontrato: 

\begin{figure}[H]
    \centering
    \includegraphics[width=\textwidth]{output/MGCvsMG-pugile-enph.png}
    \caption{Gradiente immagine fotografica}
    \label{fig:errorOutput}
\end{figure}

