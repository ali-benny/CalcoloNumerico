\documentclass{article}

% Language setting
\usepackage[italian]{babel}

% Set page size and margins
\usepackage[a4paper,top=2cm,bottom=2cm,left=2cm,right=2cm,marginparwidth=1.75cm]{geometry}

% Useful packages
\usepackage{amsmath}
\usepackage{amssymb} 
\usepackage{comment}    %scrivere commenti multirighe
\usepackage{graphicx, wrapfig}
\usepackage{microtype}
\usepackage{xcolor}
\definecolor{bblue}{HTML}{3F88C5}
\definecolor{ggreen}{HTML}{44BBA4}
\definecolor{oorange}{HTML}{FC814A}
\definecolor{rred}{HTML}{D00000}
\usepackage[colorlinks=true,allcolors=bblue]{hyperref}
\usepackage{float, caption, subcaption}

% set san-serif font for all the document
\renewcommand{\familydefault}{\sfdefault}
% text style to create a code snippet
\definecolor{codegray}{gray}{0.9}
\newcommand{\code}[1]{\colorbox{codegray}{\texttt{#1}}}
\DeclareCaptionLabelFormat{andtable}{#1~#2  \&  \tablename~\thetable}

% Title
\title{\textbf{Relazione Progetto di Calcolo Numerico} \\ \textit{"Deblur Immagini"}}
\author{Benatti Alice, Manuelli Matteo, Qayyum Shahbaz Ali}
\date{Gennaio 2022}

\begin{document}
\maketitle
\begin{figure}[b]
    \centering
    \includegraphics[width=0.6\textwidth]{imgRel/alma-logo.jpg}
\end{figure}
\newpage

% Summary
\tableofcontents
\newpage
%\hrule

% All content
\begin{comment}
Relazione

1. Riportare e commentare i risultati ottenuti nei punti 2. 3. (e 4.) 
su un immagine del set creato e su altre due immagini in bianco e nero 
(fotografiche/mediche/astronomiche)
2. Riportare delle tabelle con le misure di PSNR e MSE ottenute al 
variare dei parametri (dimensione kernel, valore di sigma, la 
deviazione standard del rumore, il parametro di regolarizzazione). 
3. Calcolare sull’intero set di immagini medie e deviazione standard 
delle metriche per alcuni valori fissati dei parametri.  
4. Analizzare su 2 esecuzioni le proprietà dei metodi numerici 
utilizzati (gradiente coniugato e gradiente) in termini di numero di 
iterazioni, andamento dell’errore, della funzione obiettivo, norma del 
gradiente. 
\end{comment}

\section{Presentazione del problema}
Il progetto ha come scopo quello di comprendere e mettere in atto metodi per ricostruire 
immagini blurrate e svolgere il lavoro opposto, quindi generare immagini corrotte dal rumore 
a partire da un immagine originale. 

Il problema che ci è stato presentato riguarda la ricostruzione di 
immagini corrotte attraverso il blur Gaussiano.

Verrà analizzata inizialmente l'immagine \code{data.camera()} importata da
\code{skimage}, successivamente verranno analizzate un set di 8 immagini con oggetti geometrici
 di colore uniforme su sfondo nero, realizzate da noi.

Il problema di deblur consiste nella ricostruzione di un immagine a partire da un dato acquisito
 mediante il seguente modello:
\[b=Ax+\eta\]
dove $b$ rappresenta l'immagine corrotta, $x$ l'immagine originale che vogliamo ricostruire, $A$ 
l'operatore che applica il blur Gaussiano ed $\eta$ il rumore additivo con distribuzione Gaussiana di
 media $\mathbb{0}$ e deviazione standard $\sigma$.

Per svolgere il progetto si farà uso dei moduli \code{numpy}, \code{skimage} e \code{matplotlib}
utilizzando il linguaggio Python.

Affinché risultino chiari i valori a cui andremo a riferirci nella relazione, bisogna tenere ben presente 
il significato di questi due parametri. 

\textbf{PSNR (Peak Signal to Noise Ratio):} Misura la qualità di un immagine ricostruita rispetto all'immagine 
originale, la formula per calcolarlo è la seguente: \[PSNR = log_{10}(\frac{max\;x^\ast}{\sqrt{MSE}})\]

\textbf{MSE (Mean Squared Error):}  Con la sigla ci riferiamo all'errore quadratico medio ed è così ottenuto:
 \[MSE = \sqrt[2]{\frac{\sum_{i=1}^n\sum_{j=1}(x^{\ast}_{ij}-x_{ij})}{nm}}\]

I due valori sono inversamente proposizionali, quindi più è alto il PSNR e basso l'MSE, più l'immagine sarà 
simile all'immagine originale. il PSNR dipende dall'MSE.

\textbf{Deviazione standard:} E' un indice che ci permette di capire in maniera riassuntiva le differenze dei 
valori per ogni osservazioni rispetto alla media delle variabili. 

    \subsection{Generazione dataset}
    {\color{bblue}\subsection{Generazione dataset}}
É richiesto un set di immagini con le seguenti specifiche: 
\begin{itemize}
    \item 8 Immagini di dimensione $512 \times 512$;
    \item Formato PNG in scala di grigi;
    \item Devono contenere tra i 2 ed i 6 oggetti geometrici;
    \item Oggetti di colore uniforme su uno sfondo nero.
\end{itemize}

\begin{figure}[H]
    \centering
    \includegraphics[width=0.5\linewidth]{./imgRel/dataset.png}\label{fig:datasetgeometriche}
    \caption{Immagini geometriche studiate}
\end{figure}

Inoltre useremo immagini "fotografiche" con dimensione $512 \times 512$, le quali verranno mostrate usando la flag 
\verb|as_gray=True| per 
poterle visualizzare in bianco e nero.

Le immagini selezionate sono le seguenti:
\begin{description}
    \item[Immagine Fotografica] Inquadra un fotografo nell'intento di uno scatto con sfondo 
    paesaggistico. (Importate all'interno del progetto con la libreria \code{skimage})
    \item[Immagine Ritratto] Ritrae il volto di una persona in modo dettagliato e con 
    varie tonalità di grigio. É necessario caricare l'immagine nel progetto con il comando \verb|??|
    per importarla all'interno del progetto e poterla analizzare.
\end{description}

\begin{figure}[H]
    \centering
    \begin{subfigure}{0.5\textwidth}
        \centering
        \includegraphics[width=0.5\linewidth]{./img/datacamera.png}\label{fig:giornale}
        \subcaption{Immagine fotografica}
    \end{subfigure}\hfill
    \begin{subfigure}{0.5\textwidth}
        \centering
        \includegraphics[width=0.5\linewidth]{./img/pugile.png}\label{fig:pugile}
        \subcaption{Immagine ritratto}
    \end{subfigure}
    
    \caption{Immagini fotografiche analizzate}
\end{figure}

    \subsection{Generazione Immagini Corrotte}
\textbf{Obiettivo:}
Degradare le immagini applicando, mediante le funzioni riportate \\lla cella precedente,  l'operatore di blur con parametri
\begin{itemize}
    \item{$\sigma=0.5$ dimensione $5\times 5$}
    \item{$\sigma=1$ dimensione $7\times 7$}
    \item{$\sigma=1.3$ dimensione $9\times 9$}
\end{itemize}
ed aggiunge rumore gaussiano con deviazione standard (0, 0.05)

\begin{figure}[H]
    \centering
    \begin{minipage}[h]{\textwidth}
    \includegraphics[width=\linewidth]{output/tabCorrotte/imgcorr1.png}\label{fig:imgcorrotte1}
    \end{minipage}
    \begin{minipage}[h]{\textwidth}
        \centering
        
        \begin{tabular}{|l c c c c r|}
            \hline
            \multicolumn{1}{|c}{\textbf{Nome Img}} & \multicolumn{1}{|c}{\textbf{DimKer}} & \multicolumn{1}{|c}{\textbf{Sigma}} & \multicolumn{1}{|c}{\textbf{Noise Dev}} & \multicolumn{1}{|c}{\textbf{PSNR}} & \multicolumn{1}{|c|}{\textbf{MSE}} \\ \hline
                img1.png & 5 & 0.5 & 0.02 & 29.9455 & 0.00101263 \\
                img2.png & 5 & 0.5 & 0.02 & 31.0498 & 0.000785276\\
                img3.png & 5 & 0.5 & 0.02 & 30.3053 & 0.000932107\\
                img4.png & 5 & 0.5 & 0.02 & 31.3947 & 0.000725313\\
                img5.png & 5 & 0.5 & 0.02 & 31.2456 & 0.00075065 \\
                img6.png & 5 & 0.5 & 0.02 & 30.9971 & 0.00079486 \\
                img7.png & 5 & 0.5 & 0.02 & 30.7187 & 0.000847478\\
                img8.png & 5 & 0.5 & 0.02 & 29.8802 & 0.00102797 \\
                pugile.png & 5 & 0.5 & 0.02 & 32.3195 & 0.0005862\\
                giornale.png & 5 & 0.5 & 0.02 & 23.7827 & 0.00418534 \\ \hline
        \end{tabular}\label{tab:tabcorrotte1}
    \end{minipage}
    \captionlistentry[table]{Table corrotte}
    \captionsetup{labelformat=andtable}
    \caption{Immagini corrotte con $\sigma = 0.5$ dimensione $5 \times 5$ e noise=0.02}
\end{figure}

\begin{figure}[H]
    \centering
    \begin{minipage}[h]{\textwidth}
    \includegraphics[width=\linewidth]{output/tabCorrotte/imgcorr2.png}\label{fig:imgcorrotte2}
    \end{minipage}
    \begin{minipage}[h]{\textwidth}
        \centering
        
        \begin{tabular}{|l c c c c r|}
            \hline
            \multicolumn{1}{|c}{\textbf{Nome Img}} & \multicolumn{1}{|c}{\textbf{DimKer}} & \multicolumn{1}{|c}{\textbf{Sigma}} & \multicolumn{1}{|c}{\textbf{Noise Dev}} & \multicolumn{1}{|c}{\textbf{PSNR}} & \multicolumn{1}{|c|}{\textbf{MSE}} \\ \hline
                img1.png & 5 & 0.5 & 0.04 & 26.5452 & 0.00221553 \\
                img2.png & 5 & 0.5 & 0.04 & 27.0007 & 0.00199492 \\
                img3.png & 5 & 0.5 & 0.04 & 26.6982 & 0.00213887 \\
                img4.png & 5 & 0.5 & 0.04 & 27.1549 & 0.00192537 \\
                img5.png & 5 & 0.5 & 0.04 & 27.0846 & 0.00195677 \\
                img6.png & 5 & 0.5 & 0.04 & 26.9876 & 0.00200098 \\
                img7.png & 5 & 0.5 & 0.04 & 26.8806 & 0.00205086 \\
                img8.png & 5 & 0.5 & 0.04 & 26.5063 & 0.00223545 \\
                pugile.png & 5 & 0.5 & 0.04 & 27.4667 & 0.0017919\\
                giornale.png & 5 & 0.5 & 0.04 & 22.7045 & 0.0053647 \\ \hline
            \end{tabular}\label{tab:tabcorrotte2}
        
        \end{minipage}
    \captionlistentry[table]{Table corrotte}
    \captionsetup{labelformat=andtable}
    \caption{Immagini corrotte con $\sigma = 0.5$ dimensione $5 \times 5$ e noise=0.04}
\end{figure}

\begin{figure}[H]
    \centering
    \begin{minipage}[h]{\textwidth}
    \includegraphics[width=\linewidth]{output/tabCorrotte/imgcorr3.png}\label{fig:imgcorrotte3}
    \end{minipage}
    \begin{minipage}[h]{\textwidth}
        \centering
        
        \begin{tabular}{|l c c c c r|}
            \hline
            \multicolumn{1}{|c}{\textbf{Nome Img}} & \multicolumn{1}{|c}{\textbf{DimKer}} & \multicolumn{1}{|c}{\textbf{Sigma}} & \multicolumn{1}{|c}{\textbf{Noise Dev}} & \multicolumn{1}{|c}{\textbf{PSNR}} & \multicolumn{1}{|c|}{\textbf{MSE}} \\ \hline
                img1.png & 5 & 0.5 & 0.08 & 21.5332 & 0.00702558 \\
                img2.png & 5 & 0.5 & 0.08 & 21.6702 & 0.00680731 \\
                img3.png & 5 & 0.5 & 0.08 & 21.5736 & 0.00696048 \\
                img4.png & 5 & 0.5 & 0.08 & 21.7089 & 0.00674704 \\
                img5.png & 5 & 0.5 & 0.08 & 21.6929 & 0.00677191 \\
                img6.png & 5 & 0.5 & 0.08 & 21.6626 & 0.00681931 \\
                img7.png & 5 & 0.5 & 0.08 & 21.633 & 0.00686601 \\
                img8.png & 5 & 0.5 & 0.08 & 21.519 & 0.00704858 \\
                pugile.png & 5 & 0.5 & 0.08 & 21.7986 & 0.0066091\\
                giornale.png & 5 & 0.5 & 0.08 & 19.9035 & 0.0102246 \\ \hline
            \end{tabular}v\label{tab:tabcorrotte3}
        
        \end{minipage}
    \captionlistentry[table]{Table corrotte}
    \captionsetup{labelformat=andtable}
    \caption{Immagini corrotte con $\sigma = 0.5$ dimensione $5 \times 5$ e noise=0.08}
\end{figure}

\begin{figure}[H]
    \centering
    \begin{minipage}[h]{\textwidth}
    \includegraphics[width=\linewidth]{output/tabCorrotte/imgcorr4.png}\label{fig:imgcorrotte4}
    \end{minipage}
    \begin{minipage}[h]{\textwidth}
        \centering
        
        \begin{tabular}{|l c c c c r|}
            \hline
            \multicolumn{1}{|c}{\textbf{Nome Img}} & \multicolumn{1}{|c}{\textbf{DimKer}} & \multicolumn{1}{|c}{\textbf{Sigma}} & \multicolumn{1}{|c}{\textbf{Noise Dev}} & \multicolumn{1}{|c}{\textbf{PSNR}} & \multicolumn{1}{|c|}{\textbf{MSE}} \\ \hline
                img1.png & 5 & 0.5 & 0.16 & 15.8162 & 0.0262048 \\
                img2.png & 5 & 0.5 & 0.16 & 15.8543 & 0.025976 \\
                img3.png & 5 & 0.5 & 0.16 & 15.8278 & 0.0261347 \\
                img4.png & 5 & 0.5 & 0.16 & 15.8649 & 0.0259124 \\
                img5.png & 5 & 0.5 & 0.16 & 15.8618 & 0.0259313 \\
                img6.png & 5 & 0.5 & 0.16 & 15.8537 & 0.0259797 \\
                img7.png & 5 & 0.5 & 0.16 & 15.849 & 0.0260077 \\
                img8.png & 5 & 0.5 & 0.16 & 15.8161 & 0.0262051 \\
                pugile.png & 5 & 0.5 & 0.16 & 15.89 & 0.0257632 \\
                giornale.png & 5 & 0.5 & 0.16 & 15.3339 & 0.0292829 \\ \hline
            \end{tabular}\label{tab:tabcorrotte4}
        
        \end{minipage}
    \captionlistentry[table]{Table corrotte}
    \captionsetup{labelformat=andtable}
    \caption{Immagini corrotte con $\sigma = 0.5$ dimensione $5 \times 5$ e noise=0.16}
\end{figure}

\begin{figure}[H]
    \centering
    \begin{minipage}[h]{\textwidth}
    \includegraphics[width=\linewidth]{output/tabCorrotte/imgcorr5.png}\label{fig:imgcorrotte5}
    \end{minipage}
    \begin{minipage}[h]{\textwidth}
        \centering
        
        \begin{tabular}{|l c c c c r|}
            \hline
            \multicolumn{1}{|c}{\textbf{Nome Img}} & \multicolumn{1}{|c}{\textbf{DimKer}} & \multicolumn{1}{|c}{\textbf{Sigma}} & \multicolumn{1}{|c}{\textbf{Noise Dev}} & \multicolumn{1}{|c}{\textbf{PSNR}} & \multicolumn{1}{|c|}{\textbf{MSE}} \\ \hline
                img1.png & 5 & 0.5 & 0.32 & 9.8721 & 0.102989 \\                 
                img2.png & 5 & 0.5 & 0.32 & 9.88127 & 0.102772 \\ 
                img3.png & 5 & 0.5 & 0.32 & 9.87516 & 0.102916 \\
                img4.png & 5 & 0.5 & 0.32 & 9.8862 & 0.102655 \\                 
                img5.png & 5 & 0.5 & 0.32 & 9.88405 & 0.102706 \\ 
                img6.png & 5 & 0.5 & 0.32 & 9.88154 & 0.102765 \\
                img7.png & 5 & 0.5 & 0.32 & 9.88 & 0.102802 \\              
                img8.png & 5 & 0.5 & 0.32 & 9.87102 & 0.103014 \\ 
                pugile.png & 5 & 0.5 & 0.32 & 9.89115 & 0.102538 \\
                giornale.png & 5 & 0.5 & 0.32 & 9.73853 & 0.10620 \\ \hline
            \end{tabular}\label{tab:tabcorrotte5}  
        
        \end{minipage}
    \captionlistentry[table]{Table corrotte}
    \captionsetup{labelformat=andtable}
    \caption{Immagini corrotte con $\sigma = 0.5$ dimensione $5 \times 5$ e noise=0.32}
\end{figure}

\begin{figure}[H]
    \centering
    \begin{minipage}[h]{\textwidth}
    \includegraphics[width=\linewidth]{output/tabCorrotte/imgcorr6.png}\label{fig:imgcorrotte6}
    \end{minipage}
    \begin{minipage}[h]{\textwidth}
        \centering
        
        \begin{tabular}{|l c c c c r|}
            \hline
            \multicolumn{1}{|c}{\textbf{Nome Img}} & \multicolumn{1}{|c}{\textbf{DimKer}} & \multicolumn{1}{|c}{\textbf{Sigma}} & \multicolumn{1}{|c}{\textbf{Noise Dev}} & \multicolumn{1}{|c}{\textbf{PSNR}} & \multicolumn{1}{|c|}{\textbf{MSE}} \\ \hline
                img1.png & 7 & 1 & 0.02 & 28.3054 & 0.00147726 \\
                img2.png & 7 & 1 & 0.02 & 29.7038 & 0.00107057 \\
                img3.png & 7 & 1 & 0.02 & 28.5689 & 0.00139029 \\
                img4.png & 7 & 1 & 0.02 & 30.0656 & 0.000985011 \\
                img5.png & 7 & 1 & 0.02 & 29.9064 & 0.0010218 \\                 
                img6.png & 7 & 1 & 0.02 & 29.6228 & 0.00109074 \\ 
                img7.png & 7 & 1 & 0.02 & 29.2918 & 0.00117711 \\
                img8.png & 7 & 1 & 0.02 & 28.2434 & 0.00149851 \\
                pugile.png & 7 & 1 & 0.02 & 30.9284 & 0.000807527 \\
                giornale.png & 7 & 1 & 0.02 & 20.9194 & 0.0080921 \\ \hline
            \end{tabular}\label{tab:tabcorrotte6}
        
        \end{minipage}
    \captionlistentry[table]{Table corrotte}
    \captionsetup{labelformat=andtable}
    \caption{Immagini corrotte con $\sigma = 1$ dimensione $7 \times 7$ e noise=0.02}
\end{figure}

\begin{figure}[H]
    \centering
    \begin{minipage}[h]{\textwidth}
    \includegraphics[width=\linewidth]{output/tabCorrotte/imgcorr7.png}\label{fig:imgcorrotte7}
    \end{minipage}
    \begin{minipage}[h]{\textwidth}
        \centering
        
        \begin{tabular}{|l c c c c r|}
            \hline
            \multicolumn{1}{|c}{\textbf{Nome Img}} & \multicolumn{1}{|c}{\textbf{DimKer}} & \multicolumn{1}{|c}{\textbf{Sigma}} & \multicolumn{1}{|c}{\textbf{Noise Dev}} & \multicolumn{1}{|c}{\textbf{PSNR}} & \multicolumn{1}{|c|}{\textbf{MSE}} \\ \hline
                img1.png & 7 & 1 & 0.04 & 25.7196 & 0.00267942 \\
                img2.png & 7 & 1 & 0.04 & 26.4282 & 0.00227604 \\
                img3.png & 7 & 1 & 0.04 & 25.8439 & 0.00260379 \\
                img4.png & 7 & 1 & 0.04 & 26.5898 & 0.0021929 \\                
                img5.png & 7 & 1 & 0.04 & 26.5209 & 0.00222797 \\ 
                img6.png & 7 & 1 & 0.04 & 26.4067 & 0.00228732 \\
                img7.png & 7 & 1 & 0.04 & 26.2277 & 0.00238357 \\
                img8.png & 7 & 1 & 0.04 & 25.6828 & 0.00270224 \\
                pugile.png & 7 & 1 & 0.04 & 26.9696 & 0.00200928 \\
                giornale.png & 7 & 1 & 0.04 & 20.3045 & 0.0093227\\ \hline
            \end{tabular}\label{tab:tabcorrotte7}
        
        \end{minipage}
    \captionlistentry[table]{Table corrotte}
    \captionsetup{labelformat=andtable}
    \caption{Immagini corrotte con $\sigma = 1$ dimensione $7 \times 7$ e noise=0.04}
\end{figure}

\begin{figure}[H]
    \centering
    \begin{minipage}[h]{\textwidth}
    \includegraphics[width=\linewidth]{output/tabCorrotte/imgcorr8.png}\label{fig:imgcorrotte7x70.08}
    \end{minipage}
    \begin{minipage}[h]{\textwidth}
        \centering
        
        \begin{tabular}{|l c c c c r|}
            \hline
            \multicolumn{1}{|c}{\textbf{Nome Img}} & \multicolumn{1}{|c}{\textbf{DimKer}} & \multicolumn{1}{|c}{\textbf{Sigma}} & \multicolumn{1}{|c}{\textbf{Noise Dev}} & \multicolumn{1}{|c}{\textbf{PSNR}} & \multicolumn{1}{|c|}{\textbf{MSE}} \\ \hline
                img1.png & 7 & 1 & 0.08 & 21.2586 & 0.00748406 \\
                img2.png & 7 & 1 & 0.08 & 21.5025 & 0.00707543 \\
                img3.png & 7 & 1 & 0.08 & 21.3156 & 0.0073866 \\                 
                img4.png & 7 & 1 & 0.08 & 21.5582 & 0.00698523 \\
                img5.png & 7 & 1 & 0.08 & 21.5463 & 0.00700433 \\
                img6.png & 7 & 1 & 0.08 & 21.5043 & 0.00707252 \\
                img7.png & 7 & 1 & 0.08 & 21.4429 & 0.00717307 \\
                img8.png & 7 & 1 & 0.08 & 21.2464 & 0.00750518 \\
                pugile.png & 7 & 1 & 0.08 & 21.6733 & 0.00680245 \\
                giornale.png & 7 & 1 & 0.08 & 18.5164 & 0.0140722 \\ \hline
            \end{tabular}\label{tab:tabcorrotte7x70.08}
        \end{minipage}
    \captionlistentry[table]{Table corrotte}
    \captionsetup{labelformat=andtable}
    \caption{Immagini corrotte con $\sigma = 1$ dimensione $7 \times 7$ e noise=0.08}
\end{figure}

\begin{figure}[H]
    \centering
    \begin{minipage}[h]{\textwidth}
    \includegraphics[width=\linewidth]{output/tabCorrotte/imgcorr9.png}\label{fig:imgcorrotte7x70.16}
    \end{minipage}
    \begin{minipage}[h]{\textwidth}
        \centering
        
        \begin{tabular}{|l c c c c r|}
            \hline
            \multicolumn{1}{|c}{\textbf{Nome Img}} & \multicolumn{1}{|c}{\textbf{DimKer}} & \multicolumn{1}{|c}{\textbf{Sigma}} & \multicolumn{1}{|c}{\textbf{Noise Dev}} & \multicolumn{1}{|c}{\textbf{PSNR}} & \multicolumn{1}{|c|}{\textbf{MSE}} \\ \hline
                img1.png & 7 & 1 & 0.16 & 15.7444 & 0.0266415 \\
                img2.png & 7 & 1 & 0.16 & 15.8054 & 0.0262698 \\ 
                img3.png & 7 & 1 & 0.16 & 15.7565 & 0.0265673 \\                 
                img4.png & 7 & 1 & 0.16 & 15.8194 & 0.0261856 \\
                img5.png & 7 & 1 & 0.16 & 15.8151 & 0.0262111 \\                 
                img6.png & 7 & 1 & 0.16 & 15.8029 & 0.0262853 \\
                img7.png & 7 & 1 & 0.16 & 15.7926 & 0.0263473 \\                 
                img8.png & 7 & 1 & 0.16 & 15.7385 & 0.0266776 \\
                pugile.png & 7 & 1 & 0.16 & 15.8496 & 0.0260039 \\
                giornale.png & 7 & 1 & 0.16 & 14.7736 & 0.0333153\\ \hline
            \end{tabular}\label{tab:tabcorrotte7x70.16}   
        
        \end{minipage}
    \captionlistentry[table]{Table corrotte}
    \captionsetup{labelformat=andtable}
    \caption{Immagini corrotte con $\sigma = 1$ dimensione $7 \times 7$ e noise=0.16}
\end{figure}

\begin{figure}[H]
    \centering
    \begin{minipage}[h]{\textwidth}
    \includegraphics[width=\linewidth]{output/tabCorrotte/imgcorr10.png}\label{fig:imgcorrotte7x70.32}
    \end{minipage}
    \begin{minipage}[h]{\textwidth}
        \centering
        
        \begin{tabular}{|l c c c c r|}
            \hline
            \multicolumn{1}{|c}{\textbf{Nome Img}} & \multicolumn{1}{|c}{\textbf{DimKer}} & \multicolumn{1}{|c}{\textbf{Sigma}} & \multicolumn{1}{|c}{\textbf{Noise Dev}} & \multicolumn{1}{|c}{\textbf{PSNR}} & \multicolumn{1}{|c|}{\textbf{MSE}} \\ \hline
                img1.png & 7 & 1 & 0.32 & 9.84505 & 0.103632 \\ 
                img2.png & 7 & 1 & 0.32 & 9.86765 & 0.103094 \\
                img3.png & 7 & 1 & 0.32 & 9.85214 & 0.103463 \\
                img4.png & 7 & 1 & 0.32 & 9.86913 & 0.103059 \\
                img5.png & 7 & 1 & 0.32 & 9.869 & 0.103062 \\
                img6.png & 7 & 1 & 0.32 & 9.86252 & 0.103216 \\ 
                img7.png & 7 & 1 & 0.32 & 9.85775 & 0.10333 \\
                img8.png & 7 & 1 & 0.32 & 9.84472 & 0.10364 \\ 
                pugile.png & 7 & 1 & 0.32 & 9.87496 & 0.102921 \\
                giornale.png & 7 & 1 & 0.32 & 9.57992 & 0.110156 \\\hline
            \end{tabular}\label{tab:tabcorrotte7x70.32}   
        
        \end{minipage}
    \captionlistentry[table]{Table corrotte}
    \captionsetup{labelformat=andtable}
    \caption{Immagini corrotte con $\sigma = 1$ dimensione $7 \times 7$ e noise=0.32}
\end{figure}

\begin{figure}[H]
    \centering
    \begin{minipage}[h]{\textwidth}
    \includegraphics[width=\linewidth]{output/tabCorrotte/imgcorr11.png}\label{fig:imgcorrotte9x90.02}
    \end{minipage}
    \begin{minipage}[h]{\textwidth}
        \centering
        
        \begin{tabular}{|l c c c c r|}
            \hline
            \multicolumn{1}{|c}{\textbf{Nome Img}} & \multicolumn{1}{|c}{\textbf{DimKer}} & \multicolumn{1}{|c}{\textbf{Sigma}} & \multicolumn{1}{|c}{\textbf{Noise Dev}} & \multicolumn{1}{|c}{\textbf{PSNR}} & \multicolumn{1}{|c|}{\textbf{MSE}} \\ \hline
                img1.png & 9 & 1.3 & 0.02 & 27.7886 & 0.00166394 \\
                img2.png & 9 & 1.3 & 0.02 & 29.2539 & 0.00118745 \\
                img3.png & 9 & 1.3 & 0.02 & 28.0014 & 0.00158437 \\
                img4.png & 9 & 1.3 & 0.02 & 29.6382 & 0.00108688 \\
                img5.png & 9 & 1.3 & 0.02 & 29.4566 & 0.00113329 \\
                img6.png & 9 & 1.3 & 0.02 & 29.1727 & 0.00120986 \\
                img7.png & 9 & 1.3 & 0.02 & 28.8132 & 0.00131424 \\
                img8.png & 9 & 1.3 & 0.02 & 27.7142 & 0.00169269 \\
                pugile.png & 9 & 1.3 & 0.02 & 30.3576 & 0.0009209\\
                giornale.png & 9 & 1.3 & 0.02 & 20.0318 & 0.00992697 \\ \hline
            \end{tabular}\label{tab:tabcorrotte9x90.02}
        
        \end{minipage}
    \captionlistentry[table]{Table corrotte}
    \captionsetup{labelformat=andtable}
    \caption{Immagini corrotte con $\sigma = 1.3$ dimensione $9 \times 9$ e noise=0.02}
\end{figure}

\begin{figure}[H]
    \centering
    \begin{minipage}[h]{\textwidth}
    \includegraphics[width=\linewidth]{output/tabCorrotte/imgcorr12.png}\label{fig:imgcorrotte9x90.04}
    \end{minipage}
    \begin{minipage}[h]{\textwidth}
        \centering
        
        \begin{tabular}{|l c c c c r|}
            \hline
            \multicolumn{1}{|c}{\textbf{Nome Img}} & \multicolumn{1}{|c}{\textbf{DimKer}} & \multicolumn{1}{|c}{\textbf{Sigma}} & \multicolumn{1}{|c}{\textbf{Noise Dev}} & \multicolumn{1}{|c}{\textbf{PSNR}} & \multicolumn{1}{|c|}{\textbf{MSE}} \\ \hline
                img1.png & 9 & 1.3 & 0.04 & 25.4375 & 0.00285925 \\
                img2.png & 9 & 1.3 & 0.04 & 26.2124 & 0.00239202 \\
                img3.png & 9 & 1.3 & 0.04 & 25.5474 & 0.0027878 \\
                img4.png & 9 & 1.3 & 0.04 & 26.389 & 0.00229666 \\
                img5.png & 9 & 1.3 & 0.04 & 26.2972 & 0.00234571 \\
                img6.png & 9 & 1.3 & 0.04 & 26.1696 & 0.00241568 \\
                img7.png & 9 & 1.3 & 0.04 & 25.9907 & 0.00251728 \\
                img8.png & 9 & 1.3 & 0.04 & 25.3782 & 0.00289853 \\
                pugile.png & 9 & 1.3 & 0.04 & 26.7366 & 0.00212 \\
                giornale.png & 9 & 1.3 & 0.04 & 19.5402 & 0.0111168 \\ \hline
            \end{tabular}\label{tab:tabcorrotte9x90.04}
        
        \end{minipage}
    \captionlistentry[table]{Table corrotte}
    \captionsetup{labelformat=andtable}
    \caption{Immagini corrotte con $\sigma = 1.3$ dimensione $9 \times 9$ e noise=0.04}
\end{figure}


\begin{figure}[H]
    \centering
    \begin{minipage}[h]{\textwidth}
    \includegraphics[width=\linewidth]{output/tabCorrotte/imgcorr13.png}\label{fig:imgcorrotte9x90.08}
    \end{minipage}
    \begin{minipage}[h]{\textwidth}
        \centering
        
        \begin{tabular}{|l c c c c r|}
            \hline
            \multicolumn{1}{|c}{\textbf{Nome Img}} & \multicolumn{1}{|c}{\textbf{DimKer}} & \multicolumn{1}{|c}{\textbf{Sigma}} & \multicolumn{1}{|c}{\textbf{Noise Dev}} & \multicolumn{1}{|c}{\textbf{PSNR}} & \multicolumn{1}{|c|}{\textbf{MSE}} \\ \hline
                img1.png & 9 & 1.3 & 0.08 & 21.1615 & 0.0076534 \\
                img2.png & 9 & 1.3 & 0.08 & 21.446 & 0.00716802 \\
                img3.png & 9 & 1.3 & 0.08 & 21.2099 & 0.00756858 \\
                img4.png & 9 & 1.3 & 0.08 & 21.4931 & 0.00709073 \\
                img5.png & 9 & 1.3 & 0.08 & 21.4791 & 0.0071136 \\
                img6.png & 9 & 1.3 & 0.08 & 21.436 & 0.0071845 \\
                img7.png & 9 & 1.3 & 0.08 & 21.3758 & 0.00728476 \\
                img8.png & 9 & 1.3 & 0.08 & 21.1529 & 0.00766845 \\
                pugile.png & 9 & 1.3 & 0.08 & 21.6072 & 0.0069068\\
                giornale.png & 9 & 1.3 & 0.08 & 17.9887 & 0.0158901 \\ \hline
            \end{tabular}\label{tab:tabcorrotte9x90.08}
        
        \end{minipage}
    \captionlistentry[table]{Table corrotte}
    \captionsetup{labelformat=andtable}
    \caption{Immagini corrotte con $\sigma = 1.3$ dimensione $9 \times 9$ e noise=0.08}
\end{figure}

\begin{figure}[H]
    \centering
    \begin{minipage}[h]{\textwidth}
    \includegraphics[width=\linewidth]{output/tabCorrotte/imgcorr14.png}\label{fig:imgcorrotte9x90.16}
    \end{minipage}
    \begin{minipage}[h]{\textwidth}
        \centering
        
        \begin{tabular}{|l c c c c r|}
            \hline
            \multicolumn{1}{|c}{\textbf{Nome Img}} & \multicolumn{1}{|c}{\textbf{DimKer}} & \multicolumn{1}{|c}{\textbf{Sigma}} & \multicolumn{1}{|c}{\textbf{Noise Dev}} & \multicolumn{1}{|c}{\textbf{PSNR}} & \multicolumn{1}{|c|}{\textbf{MSE}} \\ \hline
                img1.png & 9 & 1.3 & 0.16 & 15.7044 & 0.026888 \\
                img2.png & 9 & 1.3 & 0.16 & 15.7843 & 0.0263979 \\
                img3.png & 9 & 1.3 & 0.16 & 15.7146 & 0.0268251 \\
                img4.png & 9 & 1.3 & 0.16 & 15.7966 & 0.026323 \\
                img5.png & 9 & 1.3 & 0.16 & 15.7911 & 0.0263568 \\
                img6.png & 9 & 1.3 & 0.16 & 15.7794 & 0.0264277 \\
                img7.png & 9 & 1.3 & 0.16 & 15.7632 & 0.0265265 \\
                img8.png & 9 & 1.3 & 0.16 & 15.7044 & 0.0268884 \\
                pugile.png & 9 & 1.3 & 0.16 & 15.8334 & 0.0261013\\
                giornale.png & 9 & 1.3 & 0.16 & 14.5301 & 0.0352365 \\ \hline
            \end{tabular}\label{tab:tabcorrotte9x90.16}
        
        \end{minipage}
    \captionlistentry[table]{Table corrotte}
    \captionsetup{labelformat=andtable}
    \caption{Immagini corrotte con $\sigma = 1.3$ dimensione $9 \times 9$ e noise=0.16}
\end{figure}



\begin{figure}[H]
    \centering
    \begin{minipage}[h]{\textwidth}
    \includegraphics[width=\linewidth]{output/tabCorrotte/imgcorr15.png}\label{fig:imgcorrotte9x90.32}
    \end{minipage}
    \begin{minipage}[h]{\textwidth}
        \centering
        
        \begin{tabular}{|l c c c c r|}
            \hline
            \multicolumn{1}{|c}{\textbf{Nome Img}} & \multicolumn{1}{|c}{\textbf{DimKer}} & \multicolumn{1}{|c}{\textbf{Sigma}} & \multicolumn{1}{|c}{\textbf{Noise Dev}} & \multicolumn{1}{|c}{\textbf{PSNR}} & \multicolumn{1}{|c|}{\textbf{MSE}} \\ \hline
                img1.png & 9 & 1.3 & 0.32 & 9.85454 & 0.103406 \\
                img2.png & 9 & 1.3 & 0.32 & 9.87346 & 0.102957 \\
                img3.png & 9 & 1.3 & 0.32 & 9.85875 & 0.103306 \\
                img4.png & 9 & 1.3 & 0.32 & 9.87778 & 0.102854 \\
                img5.png & 9 & 1.3 & 0.32 & 9.87599 & 0.102897 \\
                img6.png & 9 & 1.3 & 0.32 & 9.86911 & 0.10306 \\ 
                img7.png & 9 & 1.3 & 0.32 & 9.8681 & 0.103084 \\ 
                img8.png & 9 & 1.3 & 0.32 & 9.85336 & 0.103434 \\
                pugile.png & 9 & 1.3 & 0.32 & 9.88385 & 0.10271 \\
                giornale.png & 9 & 1.3 & 0.32 & 9.52564 & 0.111541 \\ \hline
            \end{tabular}\label{tab:tabcorrotte9x90.32}
        
        \end{minipage}
    \captionlistentry[table]{Table corrotte}
    \captionsetup{labelformat=andtable}
    \caption{Immagini corrotte con $\sigma = 1.3$ dimensione $9 \times 9$ e noise=0.32}
\end{figure}

\begin{figure}[H]
    \centering
    \begin{minipage}[h]{0.58\textwidth}
    \includegraphics[width=\linewidth]{output/tabCorrotte/imgcorr0.png}\label{fig:imgcorrotteGEN}
    \end{minipage}%
    \begin{minipage}[h]{0.5\textwidth}
        \centering
        \begin{tabular}{|lr|}
            \hline
            \multicolumn{2}{|c|}{\textbf{Istanza del Problema}} \\ \hline
            Media PSNR           & 28.721034907424552           \\
            Media MSE            & 0.0013616018508439164        \\
            Dev. Std. PSNR       & 0.7263716130347668           \\
            Dev. Std. MSE        & 0.00032199072780323465       \\ \hline
            \end{tabular}
    \end{minipage}
    \captionlistentry[table]{Table corrotte}
    \captionsetup{labelformat=andtable}
    \caption{Immagini corrotte}
\end{figure}

    {\color{bblue}\subsection{Osservazioni}}
Osserviamo i risultati ottenuti su un'immagine scelta casualmente del set creato e sulle due immagini 
 fotografiche aggiunte.

Ricordiamo che più è alto il valore del PSNR maggiore sarà la vicinanza dell'immagine corrotta 
all'immagine originale. 

{\color{bblue}\subsubsection{Analisi immagine geometrica}}
Analizziamo l'immagine img8.png al variare del valore $\sigma$ con \verb|noise| fissato a 0.02:

\begin{figure}[H]
    \centering
    \begin{subfigure}{0.6\textwidth}
        \centering
        \includegraphics[width=\textwidth]{imgRel/img8corrotto/img8corrotta5x5.png}
        \caption{Img8 corrotta con $\sigma = 0.5$ dimensione $5 \times 5$}
        \label{fig:8corrotto5}
    \end{subfigure}
    \begin{subfigure}{0.6\textwidth}
        \centering
        \includegraphics[width=\textwidth]{imgRel/img8corrotto/img8corrotta7x7.png}
        \caption{Img8 corrotta con $\sigma = 1$ dimensione $7\times 7$}
        \label{fig:8corrotto7}
    \end{subfigure}
    \begin{subfigure}{0.6\textwidth}
        \centering
        \includegraphics[width=\textwidth]{imgRel/img8corrotto/img8corrotta9x9.png}
        \caption{Img8 corrotta con $\sigma = 1.3$ dimensione $9 \times 9$}
        \label{fig:8corrotto9}
    \end{subfigure}
    \caption{Immagine geometrica corrotta al variare di $\sigma$}
    \label{fig:8corrotto}
\end{figure}
Le figure di sinistra rappresentano l'immagine originale, invece a destra sono riportate le immagini corrotte 
con i rispettivi valori di PSNR e MSE. 
Notiamo che all'aumentare delle dimensioni di sigma il valore di PSNR diminuisce: ciò denota un peggioramento 
della qualità dell'immagine. Infatti le immagini subiscono un affievolimento dell'intensità della scala dei 
colori e i contorni delle varie figure geometriche perdono di fermezza. 

{\color{bblue}\subsubsection{Analisi immagini fotografiche}}
Analizziamo le immagini fotografiche al variare del valore $\sigma$ con \verb|noise| fissato a 0.02:
\begin{figure}[H]
    \centering
    \begin{subfigure}{0.67\textwidth}
        \centering
        \includegraphics[width=\textwidth]{confrontiPuntoUno/5-0.5-0.02.png}
        \caption{immagini corrotte con $\sigma = 0.5$ dimensione $5 \times 5$}
        \label{fig:fotogrCorrotte5x5}
    \end{subfigure}
    \begin{subfigure}{0.67\textwidth}
        \centering
        \includegraphics[width=\textwidth]{confrontiPuntoUno/7-1-0.02.png}
        \caption{immagini corrotte con $\sigma = 1$ dimensione $7 \times 7$}
        \label{fig:fotogrCorrotte7x7}
    \end{subfigure}
    \begin{subfigure}{0.67\textwidth}
        \centering
        \includegraphics[width=\textwidth]{confrontiPuntoUno/9-1.3-0.02.png}
        \caption{immagini corrotte con $\sigma = 1.3$ dimensione $9 \times 9$}
        \label{fig:fotogrCorrotte9x9}
    \end{subfigure}
    \caption{Immagini fotografiche corrotte al variare di $\sigma$}
    \label{fig:fotogrCorrotte}
\end{figure}

Si nota un'altra volta che all'aumentare delle dimensioni di $\sigma$ diminuisce il PSNR e l'immagine perde di 
incisività (nitidezza).
In ogni caso nelle versioni corrotte, benché risultino visivamente peggiori, si riesce ancora a ben distinguere il 
soggetto in primo piano, anche se sfocato, in tutte le immagini. 

{\color{oorange}\section{Ricostruzione di un immagine rispetto una versione corrotta}}
\textcolor{oorange}{\rule[5pt]{\textwidth}{1pt}}
Una possibile ricostruzione dell'immagine originale $x$ partendo dall'immagine corrotta $b$ è la soluzione naive 
data dal minimo del seguente problema di ottimizzazione:
\[x^* = \arg\min_x \frac{1}{2} ||Ax - b||_2^2\]

Da ora in poi, come istanza del problema durante l'analisi ci baseremo sulle ricostruzioni di
 quelle immagini con parametri
$\sigma = 1.3$ dimensione $9\times 9$ \verb|noise| = 0.02.


    \subsection{Metodo del Gradiente Coniugato (naive)}
Il metodo del gradiente coniugato è un algoritmo per la risoluzione numerica di un sistema lineare la cui matrice sia simmetrica e definita positiva
 e consente di risolvere il sistema in un numero di iterazioni che e' al massimo $n$.

La funzione $f$ da minimizzare è data dalla formula
  $f(x) = \frac{1}{2} ||Ax - b||_2^2 $, il cui gradiente $\nabla f$ è dato da
$\nabla f(x) = A^TAx - A^Tb  $.

Utilizzando il metodo del gradiente coniugato implementato dalla funzione \code{minimize}
 abbiamo calcolato la soluzione naive.

    \subsection{Metodo del Gradiente}
Il metodo del gradiente è un algoritmo che calcola il vettore di minimo globale, ovvero:
Un vettore $x^{\ast}$ è un punto di minimo globale di $f(x)$ se $f(x^{\ast}) \leq f(x) \forall x \in R^n$.

Analogamente, un vettore $x^{\ast}$ è un punto di minimo globale in senso stretto di $f(x)$ 
se $f(x^{\ast}) < f(x) \forall x \in R+n \and x \neq x^{\ast}$.

\begin{figure}[H]
    \centering
    \begin{subfigure}{0.9\textwidth}
        \centering
    \includegraphics[width=0.7\textwidth]{imgRel/datasetgradiente.png}
    \caption{Immagini geometriche ripristinate con il metodo del gradiente}
    \label{fig:geomgradiente}
    \end{subfigure}

    \begin{subfigure}{0.3\textwidth}
        \centering
    \includegraphics[width=0.7\textwidth]{imgRel/fotogrmg.png}
    \caption{Immagine fotografica ripristinata con il metodo del gradiente}
    \label{fig:pugilegradiente}
    \end{subfigure}%
    \begin{subfigure}{0.3\textwidth}\centering
        \includegraphics[width=0.7\textwidth]{imgRel/giornalemg.png}
        \caption{Immagine con testo ripristinata con il metodo del gradiente}
    \end{subfigure}
\caption{Immagini analizzate ripristinate con il Metodo del Gradiente}
    \centering
    \includegraphics[width=0.7\textwidth]{imgCode/metGrad.png}
    \caption{Codice Metodo del gradiente applicato ad una singola immagine}
\end{figure}


    {\color{oorange}\subsection{Metodo del Gradiente e Metodo del Gradiente Coniugato a confronto}}

\begin{algorithm}[H]
	\caption{Metodo del Gradiente in \code{pseudocode}}\label{alg:mg}
	\begin{algorithmic}[1]
		\State $x_0 \in R^n, k=0$
        \While{$\nabla f(x_k) \neq 0$ and $k <$ maxit}\Comment{maxit è il massimo numero di iterazioni}
            \State calcolare la direzione di discesa $d_k$ (dipendente da $\nabla f(x_k)$)
            \State calcolare il passo di discesa $\alpha_k$
            \State $x_{k+1} = x_k+\alpha_kd_k$
            \State $k = k+1$
        \EndWhile
	\end{algorithmic}
\end{algorithm}

Analizziamo il comportamento del gradiente utilizzando i due metodi sull'immagine data.camera():
\begin{figure}[H]
    \centering
    \includegraphics[width=\textwidth]{output/MGCvsMG-enph.png}
    \caption{Gradiente immagine data.camera()}
    \label{fig:MGCvsMGdatacamera}
\end{figure}

Per quanto riguarda l'immagine fotografica pugile.png, abbiamo riscontrato: 

\begin{figure}[H]
    \centering
    \includegraphics[width=\textwidth]{output/MGCvsMG-pugile-enph.png}
    \caption{Gradiente immagine fotografica}
    \label{fig:MGCvsMG-pugile}
\end{figure}

Prendendo "in prestito" il problema di ottimizzazione naive,
abbiamo "registrato" su due esecuzioni distinte (la prima su data.camera() 
(Figura \ref{fig:MGCvsMGdatacamera}) e la seconda su pugile 
(Figura \ref{fig:MGCvsMG-pugile}))
il comportamento dei metodi numerici utilizzati (in termini di numero di iterazioni, andamento 
dell'errore, della funzione obiettivo e della norma del gradiente), ottentendo un andamento 
generale comune per entrambe le esecuzioni.

Abbiamo riscontrato che il metodo del gradiente con ricerca in linea esatta è più veloce a
 raggiungere un punto dove $\nabla f(x^{\star})=0$ rispetto al metodo del gradiente coniugato
 (come si può osservare nel focus delle ultime iterazioni dell'andamento della norma del gradiente).
 Per questo motivo, considerando 
 che in linea di principio \textbf{molto} generale, in entrambii metodi l'iterata successiva si calcola come:
 \[x_{k+1} = x_k - a_k \nabla f(x_k)\]
allora avverrà che nel metodo del gradiente si presenterà prima una serie di iterati molto
 vicini l'uno dall'altro che si allontanano meno velocemente dalla vera soluzione, mantenendo quindi un PSNR
 quasi costante, poiché all'iterato successivo sommiamo una quantità molto piccola dal momento
 che viene moltiplicata per $\nabla f(x)$ la quale è molto vicino a 0. 

Dall'altra parte, l'andamento della funzione obiettivo nel metodo del gradiente coniugato è 
più ripido poiché raggiunge meno velocemente un intorno dove $\nabla f(x^{\star}) = 0$ quindi si allontana di più
 dalla vera soluzione, con ripercussioni sul PSNR ed errore relativo.
{\color{ggreen}\section{Metodi di Regolarizzazione}}
\textcolor{ggreen}{\rule[5pt]{\textwidth}{1pt}}
I metodi di regolarizzazione rinunciano a trovare la soluzione esatta del 
precedente problema di ottimizzazione, ma invece calcolano la soluzione di un problema 
leggermente diverso ma meglio condizionato. Quest’ultimo viene chiamato problema 
regolarizzato o regressione.
    \subsection{Metodo di Regolarizzazione di Tikhonov}
Per ridurre gli effetti del rumore nella ricostruzione è necessario introdurre un termine di regolarizzazione di Tikhonov. 

Si considera quindi il seguente problema di ottimizzazione:

Si deve risolvere $Ax_\epsilon = b_\epsilon$ con $b_\epsilon = b+\epsilon$. 
Al posto di risolvere direttamente il sistema lineare (se il sistema è quadrato) o di minimizzare la norma 2 del residuo 
(se il sistema è rettangolare)  $||Ax_\epsilon -b_\epsilon||_2^2$, %! davvero dobbiamo mettere la formula?!

si aggiunge un vincolo di regolarità alla soluzione e si minimizza, ad esempio 
\[||Ax_\epsilon-b_\epsilon||_2^2+\gamma_\epsilon||x_\epsilon||_2^2\] che rappresenta la \textbf{forma standard} 
della regolarizzazione di Tikhonov.

Analizziamo i grafici ottenuti cercando di ridurre il rumore nella ricostruzione delle immagini del dataset. 
\begin{figure}[H]
    \centering
    \begin{subfigure}{0.5\textwidth}
        \centering
        \includegraphics[width=\textwidth]{output/PSNR/outputPSNR-img1.png}
        \caption{PSNR di img1}
        \label{fig:img1PSNR}
    \end{subfigure}\hfill
    \begin{subfigure}{0.5\textwidth}
        \centering
        \includegraphics[width=\textwidth]{output/MSE/outputMSE-img1.png}
        \caption{MSE di img1}
        \label{fig:img1MSE}
    \end{subfigure}

    \begin{subfigure}{0.5\textwidth}
        \centering
        \includegraphics[width=\textwidth]{output/PSNR/outputPSNR-img2.png}
        \caption{PSNR di img2}
        \label{fig:img2PSNR}
    \end{subfigure}\hfill
    \begin{subfigure}{0.5\textwidth}
        \centering
        \includegraphics[width=\textwidth]{output/MSE/outputMSE-img2.png}
        \caption{MSE di img2}
        \label{fig:img2MSE}
    \end{subfigure}
    \caption{Grafici andamento PSNR e MSE nella ricostruzione delle immagini con il metodo di regolarizzazione}
\end{figure}%
\begin{figure}[H]\ContinuedFloat
    \centering
    \begin{subfigure}{0.5\textwidth}
        \centering
        \includegraphics[width=\textwidth]{output/PSNR/outputPSNR-img3.png}
        \caption{PSNR di img3}
        \label{fig:img3PSNR}
    \end{subfigure}\hfill
    \begin{subfigure}{0.5\textwidth}
        \centering
        \includegraphics[width=\textwidth]{output/MSE/outputMSE-img3.png}
        \caption{MSE di img3}
        \label{fig:img3MSE}
    \end{subfigure}

    \begin{subfigure}{0.5\textwidth}
        \centering
        \includegraphics[width=\textwidth]{output/PSNR/outputPSNR-img4.png}
        \caption{PSNR di img4}
        \label{fig:img4PSNR}
    \end{subfigure}\hfill
    \begin{subfigure}{0.5\textwidth}
        \centering
        \includegraphics[width=\textwidth]{output/MSE/outputMSE-img4.png}
        \caption{MSE di img4}
        \label{fig:img4MSE}
    \end{subfigure}

    \begin{subfigure}{0.5\textwidth}
        \centering
        \includegraphics[width=\textwidth]{output/PSNR/outputPSNR-img5.png}
        \caption{PSNR di img5}
        \label{fig:img5PSNR}
    \end{subfigure}\hfill
    \begin{subfigure}{0.5\textwidth}
        \centering
        \includegraphics[width=\textwidth]{output/MSE/outputMSE-img5.png}
        \caption{MSE di img5}
        \label{fig:img5MSE}
    \end{subfigure}

    \begin{subfigure}{0.5\textwidth}
        \centering
        \includegraphics[width=\textwidth]{output/PSNR/outputPSNR-img6.png}
        \caption{PSNR di img6}
        \label{fig:img6PSNR}
    \end{subfigure}\hfill
    \begin{subfigure}{0.5\textwidth}
        \centering
        \includegraphics[width=\textwidth]{output/MSE/outputMSE-img6.png}
        \caption{MSE di img6}
        \label{fig:img6MSE}
    \end{subfigure}
    \caption{Grafici andamento PSNR e MSE nella ricostruzione delle immagini con il metodo di regolarizzazione}
\end{figure}%
\begin{figure}[H]\ContinuedFloat
    \centering
    \begin{subfigure}{0.5\textwidth}
        \centering
        \includegraphics[width=\textwidth]{output/PSNR/outputPSNR-img7.png}
        \caption{PSNR di img7}
        \label{fig:img7PSNR}
    \end{subfigure}\hfill
    \begin{subfigure}{0.5\textwidth}
        \centering
        \includegraphics[width=\textwidth]{output/MSE/outputMSE-img7.png}
        \caption{MSE di img7}
        \label{fig:img7MSE}
    \end{subfigure}

    \begin{subfigure}{0.5\textwidth}
        \centering
        \includegraphics[width=\textwidth]{output/PSNR/outputPSNR-img8.png}
        \caption{PSNR di img8}
        \label{fig:img8PSNR}
    \end{subfigure}\hfill
    \begin{subfigure}{0.5\textwidth}
        \centering
        \includegraphics[width=\textwidth]{output/MSE/outputMSE-img8.png}
        \caption{MSE di img8}
        \label{fig:img8MSE}
    \end{subfigure}
\caption{Grafici andamento PSNR e MSE nella ricostruzione delle immagini con il metodo di regolarizzazione}
\end{figure}

Per quanto riguarda le immagini fotografiche, abbiamo riscontrato un curioso errore. 

\begin{figure}[H]
    \centering
    \includegraphics[width=\textwidth]{output/outputERROR.png}
    \caption{Errore immagini fotografiche}
    \label{fig:errorOutput}
\end{figure}

Possiamo affermare che l'incremento è talmente impercettibile che non viene colto dal calcolatore poiché inferiore alla sua precisione macchina, con la regolarizzazione si ottengono comunque risultati migliori. 

    \newpage
{\color{rred}\section{Variazione Totale}}
\textcolor{rred}{\rule[5pt]{\textwidth}{1pt}}
Tramite un algoritmo possiamo recuperare immagini sfocate basandoci sulla Variazione totale partendo da una Blurring Point-Spread function di un'immagine. 

La variazione totale è definita dalla seguente formula:                                                                                

\[TV(u) = \sum_i^n{\sum_j^m{\sqrt{||\nabla u(i, j)||_2^2 + \epsilon^2}}}\]

Per calcolare il gradiente dell'immagine $\nabla u$ usiamo la funzione \code{np.gradient} che approssima la derivata per ogni pixel calcolando la differenza tra pixel adiacenti. 

I risultati sono due immagini della stessa dimensione dell'immagine in input, una che rappresenta il valore della derivata orizzontale dx e l'altra della derivata verticale dy. Il gradiente dell'immagine nel punto $(i, j)$ è quindi un vettore di due componenti, uno orizzontale contenuto in dx e uno verticale in dy.
\[x^* = \arg\min_x \frac{1}{2} ||Ax - b||_2^2 + \lambda TV(u)\] 
il cui gradiente $\nabla f$ è dato da: 
\[\nabla f(x) = (A^TAx - A^Tb)  + \lambda \nabla TV(x)\]

{\color{rred}\subsection{Variazione totale delle immagini geometriche:}}
{\color{rred}\subsubsection{Immagine geometrica img2.png}}
\begin{figure}[H]{}
    \centering
    \includegraphics[width=0.5\textwidth]{IMMAGINI_RELAZIONE/grafico2TOTVAR_riserva.png}%
    \includegraphics[width=0.5\textwidth]{IMMAGINI_RELAZIONE/proseguimentoGraficoTOTVAR2.png}
    \includegraphics[width=0.8\textwidth]{IMMAGINI_RELAZIONE/ricostruzione2TOTVAR.png}
    \caption{Immagine Corrotta, Immagine Ricostruita}
\end{figure}

{\color{rred}\subsubsection{Immagine geometrica img4.png}}
\begin{figure}[H]{}
    \centering
    \includegraphics[width=0.5\textwidth]{IMMAGINI_RELAZIONE/grafico4TOTVAR_riserva.png}%
    \includegraphics[width=0.5\textwidth]{IMMAGINI_RELAZIONE/proseguimentoGraficoTOTVAR4.png}
    \includegraphics[width=0.8\textwidth]{IMMAGINI_RELAZIONE/ricostruzione4TOTVAR.png}
    \caption{Immagine Corrotta, Immagine Ricostruita}
\end{figure}

{\color{rred}\subsubsection{Immagine geometrica img6.png}}
\begin{figure}[H]{}
    \centering
    \includegraphics[width=0.5\textwidth]{IMMAGINI_RELAZIONE/grafico6TOTVAR_riserva.png}%
    \includegraphics[width=0.5\textwidth]{IMMAGINI_RELAZIONE/proseguimentoGraficoTOTVAR6.png}
    \includegraphics[width=0.8\textwidth]{IMMAGINI_RELAZIONE/ricostruzione6TOTVAR.png}
    \caption{Immagine Corrotta, Immagine Ricostruita}
\end{figure}


{\color{rred}\subsection{Variazione totale dell'immagine ritratto:}}
\begin{figure}[H]{}
    \centering
    \includegraphics[width=0.8\textwidth]{IMMAGINI_RELAZIONE/graficoPugileTOTVAR_ERRREL&MSE.png}
    \includegraphics[width=0.8\textwidth]{IMMAGINI_RELAZIONE/graficoPugileTOTVAR_PSNR&suaMedia.png}
    \includegraphics[width=0.8\textwidth]{imgRicostruzione/ricostruzionePugile_TOTVAR_maxPSNR33.70.png}
    \caption{Immagine Originale, Immagine Corrotta, Immagine Ricostruita}
\end{figure}


{\color{rred}\subsection{Variazione totale dell'immagine con testo:}}
\begin{figure}[H]{}
    \centering
    \includegraphics[width=0.8\textwidth]{IMMAGINI_RELAZIONE/graficoGiornaleTOTVAR_ERRREL&MSE.png}
    \includegraphics[width=0.8\textwidth]{IMMAGINI_RELAZIONE/graficoGiornaleTOTVAR_PSNR&suaMedia.png}
    \includegraphics[width=0.8\textwidth]{imgRicostruzione/ricostruzioneGiornale_TOTVAR_maxPSNR23.20.png}
    \caption{Immagine Originale, Immagine Corrotta, Immagine Ricostruita}
\end{figure}


{\color{rred}\subsection{Variazione totale dell'immagine con testo e immagini:}}
\begin{figure}[H]{}
    \centering
    \includegraphics[width=0.8\textwidth]{IMMAGINI_RELAZIONE/graficoAlbumTOTVAR_ERRREL&MSE.png}
    \includegraphics[width=0.8\textwidth]{IMMAGINI_RELAZIONE/graficoAlbumTOTVAR_PSNR&suaMedia.png}
    \includegraphics[width=0.8\textwidth]{IMMAGINI_RELAZIONE/eventuale_ricostruzioneTOTVAR_Album_33.71.png}
    \caption{Immagine Originale, Immagine Corrotta, Immagine Ricostruita}
\end{figure}

\section{Conclusioni}
Dopo aver svolto un'analisi \textit{in depth} sull'"image deblurring", possiamo
 affermare che la qualità della ricostruzione naive è leggermente migliore,
 (ma in alcuni casi addirittura peggiore) dell'immagine corrotta, poiché nella
 soluzione naive si prende troppo di riferimento l'immagine blurrata deviata
 dal rumore. Infatti se si rinuncia a trovare una soluzione esatta del problema
 naive (ovvero se applichiamo una regolarizzazione),
 otteniamo risultati nettamente migliori.

 Inoltre ... -> relazione koci o Tassi XD

\end{document}