{\color{bblue}\subsection{Generazione dataset}}
É richiesto un set di immagini con le seguenti specifiche: 
\begin{itemize}
    \item 8 Immagini di dimensione $512 \times 512$;
    \item Formato PNG in scala di grigi;
    \item Devono contenere tra i 2 ed i 6 oggetti geometrici;
    \item Oggetti di colore uniforme su uno sfondo nero.
\end{itemize}

\begin{figure}[H]
    \centering
    \includegraphics[width=0.5\linewidth]{./imgRel/dataset.png}\label{fig:datasetgeometriche}
    \caption{Immagini geometriche studiate}
\end{figure}

Inoltre useremo immagini "fotografiche" con dimensione $512 \times 512$, le quali verranno mostrate usando la flag 
\verb|as_gray=True| per 
poterle visualizzare in bianco e nero.

Le immagini selezionate sono le seguenti:
\begin{description}
    \item[Immagine Fotografica] Inquadra un fotografo nell'intento di uno scatto con sfondo 
    paesaggistico. (Importate all'interno del progetto con la libreria \code{skimage})
    \item[Immagine Ritratto] Ritrae il volto di una persona in modo dettagliato e con 
    varie tonalità di grigio. É necessario caricare l'immagine nel progetto con il comando \verb|??|
    per importarla all'interno del progetto e poterla analizzare.
\end{description}

\begin{figure}[H]
    \centering
    \begin{subfigure}{0.5\textwidth}
        \centering
        \includegraphics[width=0.5\linewidth]{./img/datacamera.png}\label{fig:giornale}
        \subcaption{Immagine fotografica}
    \end{subfigure}\hfill
    \begin{subfigure}{0.5\textwidth}
        \centering
        \includegraphics[width=0.5\linewidth]{./img/pugile.png}\label{fig:pugile}
        \subcaption{Immagine ritratto}
    \end{subfigure}
    
    \caption{Immagini fotografiche analizzate}
\end{figure}
