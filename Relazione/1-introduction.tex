
Per svolgere il progetto si farà uso dei moduli \code{numpy}, \code{skimage} e \code{matplotlib}
utilizzando il linguaggio Python. 
Il progetto ha come scopo quello di comprendere e mettere in atto metodi per ricostruire 
immagini blurrate e svolgere il lavoro opposto, quindi generare immagini corrotte (dal rumore) a partire da un immagine originale. 

Il problema che ci è stato presentato riguarda la ricostruzione di 
immagini corrotte attraverso il blur Gaussiano.

Verrà analizzata inizialmente l'immagine \code{data.camera()} importata da
\code{skimage}, successivamente verranno analizzate un set di 8 immagini con oggetti geometrici di colore uniforme su sfondo nero, realizzate da noi.

Il problema di deblur consiste nella ricostruzione di un immagine a partire da un dato acquisito mediante il seguente modello:
\[b=Ax+\eta\]
dove $b$ rappresenta l'immagine corrotta, $x$ l'immagine originale che vogliamo ricostruire, $A$ l'operatore che applica il blur Gaussiano ed $\eta$ il rumore additivo con distribuzione Gaussiana di  media $\mathbb{0}$ e deviazione standard $\sigma$.
