\documentclass{article}

% Language setting
\usepackage[italian]{babel}

% Set page size and margins
\usepackage[a4paper,top=2cm,bottom=2cm,left=2cm,right=2cm,marginparwidth=1.75cm]{geometry}

% Useful packages
\usepackage{amsmath}
\usepackage{comment}
\usepackage{graphicx}
\usepackage{microtype}
\usepackage[colorlinks=true,allcolors=teal]{hyperref}
\usepackage{xcolor}

% set san-serif font for all the document
\renewcommand{\familydefault}{\sfdefault}
% text style to create a code snippet
\definecolor{codegray}{gray}{0.9}
\newcommand{\code}[1]{\colorbox{codegray}{\texttt{#1}}}

% Title
\title{\textbf{Relazione Progetto di Calcolo Numerico}}
\author{Alice Benatti, Ali, Matteo Manuelli}
\date{14 gennaio 2022}

\begin{document}
\maketitle

% Summary
\tableofcontents

\newpage
% All content
\begin{comment}
Relazione

1. Riportare e commentare i risultati ottenuti nei punti 2. 3. (e 4.) 
su un immagine del set creato e su altre due immagini in bianco e nero 
(fotografiche/mediche/astronomiche)
2. Riportare delle tabelle con le misure di PSNR e MSE ottenute al 
variare dei parametri (dimensione kernel, valore di sigma, la 
deviazione standard del rumore, il parametro di regolarizzazione). 
3. Calcolare sull’intero set di immagini medie e deviazione standard 
delle metriche per alcuni valori fissati dei parametri.  
4. Analizzare su 2 esecuzioni le proprietà dei metodi numerici 
utilizzati (gradiente coniugato e gradiente) in termini di numero di 
iterazioni, andamento dell’errore, della funzione obiettivo, norma del 
gradiente. 
\end{comment}

\section{Presentazione del problema}

Il problema che ci è stato presentato riguarda la ricostruzione di 
immagini corrotte attraverso il blur Gaussiano.

Verrà analizzata inizialmente l'immagine \code{data.camera()} importata da
\code{skimage}, successivamente verranno analizzate un set di 8 immagini con oggetti geometrici di colore uniforme su sfondo nero, realizzate da noi.

[...]

\end{document}