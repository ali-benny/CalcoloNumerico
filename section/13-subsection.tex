\subsection{Osservazioni}

Osserviamo il risultato su un'immagine scelta casualmente del set creato e sulle due immagini aggiuntive: 

La figura che analizziamo variando i valori di sigma è l'immagine numero 8. 
%! immagini
Ricordiamo che più è alto il valore del PSNR maggiore sarà la vicinanza dell'immagine corrotta rispetto alla versione originale. Le figure di sinistra rappresentano l'immagine originale, invece a destra sono riportate le immagini corrotte con i rispettivi valori di PSNR. Notiamo che all'aumentare delle dimensioni di sigma il valore di PSNR diminuisce che denota un peggioramento della qualità dell'immagine, infatti le immagini subiscono un'appiattimento dell'intensità della scala dei colori e i contorni delle varie figure geometriche perdono di fermezza. Inoltre è curioso notare..

Valutiamo ora l'immagine fotografica: 
%! immagini
Si nota un'altra volta che all'aumentare delle dimensioni di sigma diminuisce il PSNR e l'immagine perde di incisività, le versioni corrotte benché risultino visivamente peggiori, si riesce ancora a ben distinguere il soggetto in primo piano, anche se sfocato, in tutte le immagini. 

Passando alla valutazione dell'immagine con testo:
%! immagini
In questa immagine abbiamo una raccolta di prime pagine di giornale che ci permettono di osservare e valutare meglio la differenza tra l'immagine originale e la versione corrotta, per esempio con $\sigma = 0,5$ otteniamo un un immagine con del testo ancora leggibile sebbene meno nitida, la difficoltà inizia ad essere maggior invece con $\sigma = 1$ dove le scritte più piccole diventano quasi illeggibili, con $\sigma = 1.3$ il PSNR diminuisce ancora sebbene non molto rispetto rispetto a sigma uguale a 1, ma in questo caso anche le scritte più grandi, fatta eccezione per i titoli, perdono di chiarezza. 
